
\chapter{Conclusions}

%=========================================================================

\begin{synopsis}

The chapter summarises the major contributions of the thesis, including a discussion of the associated limitations and directions for further research.

\end{synopsis}

%=======================

\section{New approaches to wave identification}

The first major contribution of the thesis is the new approaches it presents for the identification and characterisation of long-lived, quasi-stationary waveforms. By adapting data analysis techniques from related fields of research (i.e. the wave envelope construct used in identifying synoptic-scale Rossby wave packets and a grid rotation method commonly used in ocean modelling) these approaches fully exploit the capabilities of Fourier analysis, thus allowing a more complete description of the waveform characteristics (e.g. wave phase and amplitude). A limitation of the approaches is the occurrence of false positives (e.g. the identification algorithm might designate a data time as displaying PSA-like variability when in fact only a single anomaly of wavenumber 5 or 6 scale was observed), however the false positive rate was likely lower than for existing grid point or EOF-based approaches (e.g. Section \ref{s:zw_spatial_characteristics}).

While the most obvious future application for these new approaches is subsequent studies of the SH zonal waves and PSA pattern, they could also be adapted for use in future studies of other quasi-stationary waveforms. The most obvious candidate is the Pacific-North American (PNA) pattern \citep{Wallace1981}, which plays an important role in winter climate variability over the North Pacific and North America \citep[e.g.][]{Notaro2006}. Like its namesake, the PNA pattern follows an approximate great circle path, has traditionally been analysed via EOF analysis and has been implicated in recent mid-to-high latitude trends \citep[e.g.][]{Ding2014,Liu2015}. Other non-zonal waveforms that do not follow an approximate great circle path would be more challenging, however methods have been developed for applying Fourier analysis to synoptic-scale, non-zonal waveforms \citep{Zimin2006,Souders2014} and may represent a starting point for further research. 

%Teng2012 suggest that there is a NH ZW3, however the approach we used might not be appropriate because it doesn't dominate the spectrum


\section{New insights into the zonally asymmetric features of the SH circulation}

The application of these identification methods revealed new insights into the fundamental characteristics of the SH zonally asymmetric circulation. In particular, it was found that while ZW1 and ZW3 are both prominent features of the climatological circulation, the defining feature of highly meridional hemispheric states is an enhancement of the ZW3 component. Only a weak relationship was identified between the PSA pattern and ENSO, suggesting that the pattern might be better conceptualised as preferred regional atmospheric response to various external (and internal) forcings. PSA-2 might not exist and is instead just the amalgamation of Amundsen sea variability.

Future work (fundamental questions): Relationship with SAM, tropical forcing. What causes variability in the ZW3?
Limitations: Short time period, data sparse high southern lats


Insights into regional climate variability
Important drivers of temperature, precipitation and sea ice variability in the mid-to-high southern latitudes. 
Identified seasonal trends towards the negative phase of the PSA pattern were consistent with recent warming observed over the Antarctic Peninsula during autumn, but were inconsistent with observed winter warming over West Antarctica. 

Further work: Still can't fully reconcile atmospheric and other drivers with high latitude trends. Variability studies need to consider ZW3.
Limitations: Trends in reanalysis data are particularly problematic. As are observations of the actual trends.


Southern Hemisphere mid-to-upper tropospheric planetary wave activity is characterised by the superposition of two zonally-oriented, quasi-stationary waveforms: zonal wavenumber one (ZW1) and zonal wavenumber three (ZW3). Previous studies have tended to consider these waveforms in isolation and with the exception of those studies relating to sea ice, little is known about their impact on regional climate variability. We take a novel approach to quantifying the combined influence of ZW1 and ZW3, using the strength of the hemispheric meridional flow as a proxy for zonal wave activity. Our methodology adapts the wave envelope construct routinely used in the identification of synoptic-scale Rossby wave packets and improves on existing approaches by allowing for variations in both wave phase and amplitude. While ZW1 and ZW3 are both prominent features of the climatological circulation, the defining feature of highly meridional hemispheric states is an enhancement of the ZW3 component. Composites of the mean surface conditions during these highly meridional, ZW3-like anomalous states (i.e. months of strong planetary wave activity) reveal large sea ice anomalies over the Amundsen and Bellingshausen Seas during autumn and along much of the East Antarctic coastline throughout the year. Large precipitation anomalies in regions of significant topography (e.g. New Zealand, Patagonia, coastal Antarctica) and anomalously warm temperatures over much of the Antarctic continent were also associated with strong planetary wave activity. The latter has potentially important implications for the interpretation of recent warming over West Antarctica and the Antarctic Peninsula.

The Pacific-South American (PSA) pattern is an important mode of climate variability in the mid-to-high southern latitudes. It is widely recognised as the primary mechanism by which the El Ni\~{n}o-Southern Oscillation (ENSO) influences the south-east Pacific and south-west Atlantic, and in recent years has also been suggested as a mechanism by which longer-term tropical sea surface temperature trends can influence the Antarctic climate. Despite this recognition, relatively little is known about its climatological characteristics. This issue is addressed here by the development and application of a novel methodology for objectively identifying the PSA pattern from ERA-Interim reanalysis data. By rotating the global coordinate system such that the equator (a great circle) traces the approximate path of the pattern, the identification algorithm utilises Fourier analysis as opposed to a traditional Empirical Orthogonal Function approach. The resulting climatology reveals that the PSA pattern has a strong influence on temperature and precipitation variability over West Antarctica and the Antarctic Peninsula, and on sea ice variability in the adjacent Amundsen, Bellingshausen and Weddell Seas. Identified seasonal trends towards the negative phase of the PSA pattern are consistent with warming observed over the Antarctic Peninsula during autumn, but are inconsistent with observed winter warming over West Antarctica. Only a weak relationship is identified between the PSA pattern and ENSO, which suggests that the pattern might be better conceptualised as preferred regional atmospheric response to various external (and internal) forcings.


\section{A practical solution to the reproducibility crisis}


Weather and climate science has undergone a computational revolution in recent decades, to the point where all modern research relies heavily on software and code. Despite this profound change in the research methods employed by weather and climate scientists, the reporting of computational results has changed very little in relevant academic journals. This lag has led to something of a reproducibility crisis, whereby it is impossible to replicate and verify most of today's published computational results. While it is tempting to simply decry the slow response of journals and funding agencies in the face of this crisis, there are very few examples of reproducible weather and climate research upon which to base new communication standards. In an attempt to address this deficiency, this essay describes a procedure for reporting computational results that was employed in a recent \textit{Journal of Climate} paper. The procedure was developed to be consistent with recommended computational best practices and seeks to minimise the time burden on authors, which has been identified as the most important barrier to publishing code. It should provide a starting point for weather and climate scientists looking to publish reproducible research, and it is proposed that journals could adopt the procedure as a minimum standard.

Future work: Lobby AMS, comprehensibility (which is also a limitation), culture change




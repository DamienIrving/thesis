
\chapter{Conclusions}\label{c:conclusions}

%=========================================================================

\begin{synopsis}
This chapter summarises the major contributions of the thesis, including a discussion of the associated limitations and directions for further research.
\end{synopsis}

%=======================

\section{New approaches to wave identification}

The first major contribution of the thesis is the new approaches it presents for the identification and characterisation of long-lived, quasi-stationary waveforms. By adapting data analysis techniques from related fields of research (i.e. the wave envelope construct used in identifying synoptic-scale Rossby wave packets and a grid rotation method commonly used in ocean modelling) these approaches fully exploit the capabilities of Fourier analysis, thus allowing a more complete description of the waveform characteristics (e.g. wave phase and amplitude). A limitation is the occurrence of false positives (e.g. the identification algorithm might designate a data time as displaying PSA-like variability when in fact only a single anomaly of wavenumber six-scale was observed), however the false positive rate was likely lower than for existing grid point or EOF-based approaches (e.g. Section \ref{s:zw_spatial_characteristics}).

While an obvious future application for these new approaches is subsequent studies of SH zonal wave activity and/or the PSA pattern, they could also be adapted for use in studies of other quasi-stationary waveforms. The most closely related waveform is the Pacific-North American (PNA) pattern \citep{Wallace1981}, which plays an important role in winter climate variability over the North Pacific and North America \citep[e.g.][]{Notaro2006}. Like its SH counterpart, the PNA pattern follows an approximate great circle path, has traditionally been analysed via EOF analysis and has been implicated in recent mid-to-high latitude trends \citep[e.g.][]{Ding2014,Liu2015}. Other non-zonal waveforms that do not follow an approximate great circle path would be more challenging, however methods have been developed for applying Fourier analysis to synoptic-scale, non-zonal waveforms \citep{Zimin2006,Souders2014} and may represent a starting point for further research. 

%Teng2012 suggest that there is a NH ZW3, however the approach we used might not be appropriate because it doesn't dominate the spectrum


\section{New insights into the zonally asymmetric features of the SH circulation}

Application of the new wave identification approaches revealed new insights into both the fundamental characteristics of the zonally asymmetric circulation and its influence on regional climate variability. With respect to the former, it was found that while ZW1 and ZW3 are both prominent features of the climatological circulation, the defining feature of highly meridional hemispheric states is an enhancement of the ZW3 component. The results also confirmed the existence of the PSA-1 mode described by previous authors, but questioned the physical reality of the PSA-2 mode. Only a weak relationship was found between the PSA pattern and ENSO, suggesting that the pattern might be better conceptualised as a preferred regional atmospheric response to various external (and internal) forcings. 

These insights highlight a number of areas for further research. Given that the PSA pattern is widely regarded as the primary mechanism by which ENSO influences the high southern latitudes, our understanding of the relationship between the pattern and tropical convection may need to be revisited. Forcing of the zonal waves is also an area that warrants further attention. While they are thought to owe their existence to the configuration of the SH land masses and significant topography \citep{Baines1989}, little is known about the drivers of zonal wave variability. The results presented here suggest that ZW3 variability would be a particularly important focus for this future work.  

Surface temperature, precipitation and sea ice were shown to vary strongly in association with the zonal waves and PSA pattern, which provides a strong case for this future research into their fundamental characteristics. The documented influence of the zonal waves on regional climate variability is also of particular relevance to future studies of Antarctic temperature and mid-to-high latitude precipitation variability, given that the waves have received scant attention in this area to date. The identified seasonal trends towards the negative phase of the PSA pattern were not consistent with recent warming over West Antarctica (and were only partially consistent with warming over the Antarctic Peninsula), which suggests that further research is still needed to fully reconcile the role of the atmosphere in recent mid-to-high latitude trends. 

As with studies concerning regional climate trends, a limitation of the zonal wave and PSA pattern climatologies presented here is the relatively short time period over which the study was conducted. Modes of climate variability such as the Interdecadal Pacific Oscillation \citep{Power1999} vary over decadal and longer timescales, which means a 36 year study period is unlikely to capture the full range of variability. The sparsity of the observational network prior to the beginning of the satellite era means it would be difficult for future studies to extend the time period back beyond 1979. In fact, even after 1979 the lack of observational data for assimilation in the mid-to-high southern latitudes is a source of uncertainty for all reanalysis products. Future studies might consider comparing results obtained from multiple latest-generation reanalysis products, in an attempt to at least partially quantify this uncertainty. 


\section{A practical solution to the reproducibility crisis}

The results presented in this thesis are completely reproducible, thanks to the addition of a brief computation section detailing the associated software and code (Section \ref{s:computation}). The procedure used for documenting these computational aspects of the work was developed to be consistent with recommended computational best practices and seeks to minimise the time burden on authors, which has been identified as the most important barrier to publishing code. It should provide a starting point for weather and climate scientists looking to publish reproducible research, and a detailed proposal has been outlined explaining how journals might adopt the procedure as a formal minimum communication standard.

With this practical solution \citep{IrvingSimmonds2015} and proposed minimum standard \citep{IrvingBAMS2016} now published in the peer reviewed literature, the next step is to lobby relevant decision makers. Since both articles were published in journals managed by the American Meteorological Society (AMS), the most obvious decision makers to target in the first instance were the AMS Board on Data Stewardship. They recently decided upon a set of dataset disclosure standards that now apply to all AMS journals \citep{Mayernik2015} and hence code is next on their agenda. At the time of writing, the proposed minimum communication standard outlined in Chapter \ref{c:reproducibility} is set to be discussed by the Board at the AMS Annual Meeting in January 2016. A number of researchers have volunteered to try and adhere to the standard when they write their next paper, and their feedback will be forwarded on. Another prominent publisher in the weather and climate sciences is the Royal Meteorological Society, so they could be approached next. Societies like the American Geophysical Union and the European Geosciences Union also manage a number of journals that publish weather and climate science, however in many cases these journals also publish work relating to other fields of geoscientific research. A question that needs to be considered in more detail is whether the proposed communication standard could be adopted in other computational sciences. If it can, then the impact of this thesis might be felt far beyond the narrow field of weather and climate science. 


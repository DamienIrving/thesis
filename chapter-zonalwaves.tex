
\chapter{Zonal wave climatology}

%=========================================================================


\section{Results}

%================================

The results begin with a summary of the spatial and temporal characteristics of SH planetary wave activity, before considering its relationship with both the major modes of SH climate variability (SAM and ENSO) and surface conditions. 


\subsection{Spatial characteristics}\label{s:spatial_characteristics}

Visual inspection of the SH circulation revealed that days (with 30-day running mean applied) of PWI greater than the 90th percentile overwhelmingly exhibit a mixed ZW1 / ZW3 hemispheric planetary wave pattern (Figure \ref{fig:pwi_spatial_summary}). Days of PWI less than the 90th percentile become increasingly unlikely to exhibit a coordinated hemispheric wave pattern, so the 90th percentile was taken as a threshold value for planetary wave activity. 

Elements of the mixed ZW1 / ZW3 pattern shown in Figure \ref{fig:pwi_spatial_summary} are not unique to days where the PWI is greater than its 90th percentile. As demonstrated in Figure \ref{fig:periodograms}a, the ZW1 component of the flow is relatively insensitive to changes in the strength of the meridional flow. Instead, it appears that the main difference between days of very strong (PWI $>$ 90th percentile) and very weak (PWI $<$ 10th percentile) meridional flow is the prominence of the ZW3 component. While influential at all times, the ZW3 component is far more prominent when the meridional flow is strong and therefore dominates the streamfunction anomaly patterns (and thus surface impacts) discussed in the section on surface conditions below. 

\begin{figure}
\begin{center}
\includegraphics[width=0.7\columnwidth]{figures/Figure3.eps}
\caption{\label{fig:pwi_spatial_summary}
Composite mean 500 hPa circulation for days (with 30-day running mean applied) where the PWI was greater than its 90th percentile. Grey streamlines indicate the direction of the composite mean wind, while the black contours show the composite mean streamfunction zonal anomaly (dashed contours indicate negative values and the contour interval is $2.5 \times 10^6 \: m^2 s^{-1}$).}
\end{center}
\end{figure}

\begin{figure}
\begin{center}
\includegraphics[width=1\columnwidth]{figures/Figure4.eps}
\caption{\label{fig:periodograms}
Temporal average (1979-2014) periodograms for the 500 hPa meridional wind at 54.75$^{\circ}$S. Panel (a) shows three different subsets of the 30-day running mean data (all data times, PWI greater than its 90th percentile, PWI less than its 10th percentile), while panel (b) includes all data times but varies the running mean (in days) applied to the data prior to the Fourier transform. The labels on the vertical axis correspond to Equation \ref{eq:variance_explained}.}
\end{center}
\end{figure}


Given the dominance of the ZW3, it is not surprising that the ZW3 index of \citet{Raphael2004} shows a reasonably high level of agreement with the PWI (Figure \ref{fig:metric_vs_zw3}). Having said that, it is important to note that the shading of the dots in Figure \ref{fig:metric_vs_zw3} --- which represent the phase of the wavenumber three component of the Fourier transform --- are not randomly distributed. Whenever the phase of the wavenumber three component of the flow does not match up with the location of the three grid points used to calculate the ZW3 index (indicated by the dark grey or near-white shading), a low value is recorded for the ZW3 index. The outlying dots in the bottom right hand quadrant are particularly noteworthy, as in these cases the PWI (and hence in most cases the amplitude of the wavenumber three component of the flow) is actually quite large. The failure of the ZW3 index to capture these out-of-phase patterns means that composite analyses based on that index may overstate the stationarity (and hence the time-mean impacts) of the ZW3 component of the flow.

\begin{figure}
\begin{center}
\includegraphics[width=0.7\columnwidth]{figures/Figure5.eps}
\caption{\label{fig:metric_vs_zw3}
PWI versus the ZW3 index of \citet{Raphael2004}. Both were calculated using 500 hPa, 30-day running mean data (the PWI was calculated from the meridional wind and the ZW3 index from the geopotential height zonal anomaly). The gray shading represents the phase of the wavenumber three component of the Fourier transform of the meridional wind (expressed as the location, in degrees east, of the first local maxima), while the black line is a linear least-squares line of best fit. Both indices have been normalized to aid visual comparison (for each index this involved subtracting the mean of the index series and then dividing by the standard deviation).}
\end{center}
\end{figure}
    
\subsection{Temporal characteristics}

Consistent with previous studies \citep[e.g.][]{vanLoon1984,Mo1985}, the composite mean 500 hPa streamfunction zonal anomaly pattern for days (with 30-day running mean applied) where the PWI exceeds its 90th percentile (Figure \ref{fig:pwi_spatial_summary}) migrates zonally by approximately 15$^{\circ}$ from its most easterly location during summer to its most westerly during winter (notwithstanding the fact that the pattern breaks down from around 240-330$^{\circ}$E during summer). It has a slightly larger amplitude during the winter months and the frequency of strong planetary wave activity was also far more pronounced at that time of the year (Figure \ref{fig:annual_distribution}a). The seasonal counts of the number of days exceeding the PWI 90th percentile (Figure \ref{fig:annual_distribution}b) show that 1980 was associated with a particularly high frequency of planetary wave activity, however there were no statistically significant linear trends in these counts for timeseries including or excluding (i.e. 1981-2014) the year 1980.  

\begin{figure}
\begin{center}
\includegraphics[width=1\columnwidth]{figures/Figure6.eps}
\caption{\label{fig:annual_distribution}
Distribution of days (with 30-day running mean applied) where the PWI was greater than its 90th percentile. Monthly totals for the entire study period (1979-2014) are shown in panel (a), while the total days for each individual season are shown in panel (b). To account for the fact that not all months have an equal number of days, the counts for each month (panel a) are presented as a percentage of the total number of days for that month. Years in panel (b) are defined from December to November (e.g. the `year' 1980 spans December 1979 to November 1980).}
\end{center}
\end{figure}

While our focus is on the monthly (30-day running mean) timescale, it is interesting to consider whether similar behaviour is observed at other timescales. It can be seen from Figure \ref{fig:periodograms}b that wavenumber three dominates the average periodogram when the running mean applied to the daily 500 hPa meridional wind is greater than 10 days, with wavenumber one becoming progressively more influential as the smoothing increases. When the same process is repeated using the 500 hPa geopotential height (not shown), the results are very different. The ZW1 dominates at all timescales and except for a slight upswing from wavenumber two to three, the variance explained monotonically decreases for subsequent wavenumbers. This is an important result because \citet{vanLoon1972} analysed geopotential height data and concluded that ZW1 explains (by an appreciable margin) the largest fraction of the spatial variance in the 500hPa SH circulation (a finding that has been quoted in many subsequent papers). In light of the results presented here and the previous discussion about the fact that $v_k \propto k Z_k$ in Fourier space and that the meridional wind may be a more appropriate quantity to analyse in this context, it is clear that ZW3 plays a greater role than previously thought, particularly when there is a strong meridional component to the hemispheric flow. 

  
\subsection{SAM and ENSO}

Composite analysis was also used to assess the relationship between the PWI and the major modes of SH climate variability. SAM events were defined according to the 75th and 25th percentiles of the AOI, while positive (El Ni\~{n}o) and negative (La Ni\~{n}a) ENSO events were defined as a Ni\~{n}o 3.4 above 0.5$^{\circ}$C and below -0.5$^{\circ}$C respectively. Composites for each phase of SAM and ENSO were then calculated by taking the average across all data times for which the PWI exceeded its 90th percentile \textit{and} the AOI or Ni\~{n}o 3.4 was greater or less than the relevant threshold. 

The SAM composites show that the phase of the planetary wave pattern moves east during positive SAM events and west during negative events (Figure \ref{fig:sam_composite}). Planetary wave activity was also more common when the SAM was negative: of the 1312 data times where the PWI exceeded its 90th percentile, 510 (39\%) had an AOI that was less than the 25th percentile as compared to only 166 (13\%) with an AOI greater than the 75th percentile. Consistent with this finding, the aforementioned outlying year for planetary wave activity (1980; see Figure \ref{fig:annual_distribution}b) was associated with a large negative SAM.

The association between ENSO and planetary wave activity was far less pronounced. Besides a subtle east (La Ni\~{n}a) / west (El Ni\~{n}o) movement of the anticyclone over the south-east Pacific, no appreciable changes were seen in the phase of the planetary wave pattern (not shown) and planetary wave activity was only slightly more common during El Ni\~{n}o conditions (282 data times to 176).      

\begin{figure}
\begin{center}
\includegraphics[width=0.56\columnwidth]{figures/Figure7.eps}
\caption{\label{fig:sam_composite}
Composite mean 500 hPa streamfunction zonal anomaly for days (with 30-day running mean applied, 1979-2014) where the PWI was greater than its 90th percentile and the AOI is greater than its 75th percentile (gray) or less than its 25th percentile (black). Dashed contours indicate negative values and the contour interval is $4.0 \times 10^6 \: m^2 s^{-1}$.}
\end{center}
\end{figure}


\subsection{Surface conditions}\label{s:surface_conditions}

In order to assess the influence of planetary wave activity on regional climate variability, composite means of variables of interest (the surface air temperature anomaly, precipitation anomaly and sea ice concentration anomaly) were calculated for all data times where the PWI exceeded its 90th percentile. In other words, we asked the question: what is the average temperature (or precipitation or sea ice concentration) anomaly when there is strong planetary wave activity? The anomalous flow associated with these composites (indicated by the 500 hPa streamfunction anomaly as opposed to the streamfunction \textit{zonal} anomaly shown earlier) has a very strong ZW3 signature. This is consistent with the spatial characteristics presented earlier, which indicate that the distinguishing feature of days of strong meridional flow is the enhanced ZW3 component (as opposed to ZW1).  


\subsubsection{Surface air temperature}

Planetary wave activity was found to be associated with large and widespread surface air temperature anomalies over and/or around much of West Antarctica during all seasons (Figure \ref{fig:tas_composite}). The most pronounced anomalies were seen during autumn and winter, with warmer than average conditions over the interior of West Antarctica (associated with an anomalous northerly flow) and correspondingly colder than average conditions over the Weddell Sea (associated with an anomalous southerly flow). Due to the aforementioned seasonal migration (and breakdown during summer) of the mean planetary wave pattern, warm anomalies were confined to the Antarctica Peninsula during spring, while summer was associated with the smallest anomalies of any season.  

With respect to other sectors of the high southern latitudes, anomalously warm temperatures were widespread over East Antarctica during all seasons except summer. The largest anomalies were seen over Wilkes Land during autumn and winter, in association with an anomalous northerly flow in that region. Other features of note included anomalously cool temperatures over the Ross Sea and mainland Australia during spring.

\begin{figure}
\begin{center}
\includegraphics[width=0.63\columnwidth]{figures/Figure8.eps}
\caption{\label{fig:tas_composite}
Composite mean surface air temperature anomaly for days (with 30-day running mean applied) where the PWI was greater than its 90th percentile. Black contours show the corresponding composite mean 500 hPa streamfunction anomaly (dashed contours indicate negative values and the contour interval is $2.5 \times 10^6 \: m^2 s^{-1}$), while the hatching shows regions where the difference between the composite mean and climatological mean is significant at the $p < 0.01$ level.}
\end{center}
\end{figure}

    
\subsubsection{Precipitation}

The largest composite mean precipitation anomalies were associated with either enhanced or suppressed flow over significant topography (Figure \ref{fig:pr_composite}). For instance, the same anomalous onshore flow that was associated with high temperatures over both West Antarctica and Wilkes Land was also associated with large positive coastal precipitation anomalies, with the precise location of those anomalies moving with seasonal variations in the location of the mean planetary wave pattern. In contrast, weakened westerly flow over the southern Andes during autumn, winter and spring (but not summer due to the breakdown of the wave pattern in that region) was associated with large negative precipitation anomalies over Chilean Patagonia. A similar mechanism explains the large negative anomalies over New Zealand during winter and spring, when the mean planetary wave pattern is located far enough to the west to exert an appreciable influence on the westerly flow over the South Island. Enhanced orographic precipitation due to anomalous onshore flow might also play a role in the large positive precipitation anomalies seen over eastern Australia during spring, however the anomalies extend far beyond the Great Dividing Range, suggesting that enhanced moisture transport from the Tasman Sea might be the dominant mechanism. 

\begin{figure}
\begin{center}
\includegraphics[width=0.63\columnwidth]{figures/Figure9.eps}
\caption{\label{fig:pr_composite}
As per Figure \ref{fig:tas_composite}, but for the precipitation anomaly.}
\end{center}
\end{figure}

    
\subsubsection{Sea ice}

The sea ice composites (Figure \ref{fig:sic_composite}) were highly consistent with the temperature composites. For instance, in autumn and winter anomalous onshore flow and warmth over the interior of West Antarctica was in accord with the reduced sea ice concentration over the Amundsen Sea, and anomalous offshore flow and cold over the Weddell Sea during autumn was consistent with the increased sea ice concentration. In spring, the anomalous warmth over the western aspect of the Antarctic Peninsula concurred with the reduced sea ice concentration in the Bellingshausen Sea, while the anomalously cold temperatures over the Ross Sea coincided with an anomalously high sea ice concentration. The anomalous onshore winds and warmth over Wilkes Land were consistent with the reduced sea ice seen immediately to the north of George V Land, Ad{\'e}lie Land and the Sabrina Coast of East Antarctica in all seasons except summer (when there is very little ice there anyway). One regional feature that does not appear to fit with this overall consistency was the anomalously high sea ice concentration to the west of the Antarctic Peninsula during autumn, which was not reflected in the corresponding temperature composite.
 
\begin{figure}
\begin{center}
\includegraphics[width=0.63\columnwidth]{figures/Figure10.eps}
\caption{\label{fig:sic_composite}
As per Figure \ref{fig:tas_composite}, but for the sea ice concentration anomaly.}
\end{center}
\end{figure} 
 

\section{Discussion}

%================================

A novel method for identifying quasi-stationary planetary wave activity has been developed and applied to the problem of characterizing the SH zonal waves and their influence on regional climate variability. The method adapts the wave envelope construct traditionally used in the identification of synoptic-scale Rossby wave packets and improves on existing methods by allowing for variations in both wave phase and amplitude. The zonal wave analysis reveals that while both ZW1 and ZW3 are prominent features of the climatological SH circulation, the defining feature of highly meridional hemispheric states is an enhancement of the ZW3 component. These enhanced ZW3 states are associated with large sea ice anomalies over the Amundsen and Bellingshausen Seas and along much of the East Antarctic coastline, large precipitation anomalies in regions of significant topography and anomalously warm temperatures over much of the Antarctic continent. 

In interpreting the results of our zonal wave analysis, it is important to clearly define what is meant by the phrase `quasi-stationary planetary wave activity' in the context of this study. It was evident from our analysis of the monthly timescale (30-day running mean) meridional wind that the ZW1 and ZW3 patterns are a prominent feature of the SH circulation even when the hemispheric meridional flow is weak (Figure \ref{fig:periodograms}a). As the meridional flow gets stronger the ZW3 component becomes increasingly prominent, while the ZW1 component remains relatively unchanged. This means the average anomalous flow associated with a highly meridional hemispheric state clearly resembles a ZW3 pattern (Figures \ref{fig:tas_composite}, \ref{fig:pr_composite} and \ref{fig:sic_composite}). It is this highly meridional and anomalous ZW3 circulation that is captured by high values of our PWI and thus we refer to as planetary wave activity, as opposed to the mixed ZW1 / ZW3 pattern that is essentially present at all times.  

Our climatology of planetary wave activity confirms previous results regarding the seasonality of the zonal waves (peak activity in winter, seasonal migration of the zonal location/phase), and also identifies a large sector of the western hemisphere (120-30$^{\circ}$W) where the mean wave activity breaks down during summer. In contrast to the results presented here, previous studies have suggested a link between planetary wave activity and ENSO \citep[e.g.][]{Trenberth1980,Raphael2003,Hobbs2007}. Given the hemispheric nature of the PWI (i.e. it responds most strongly to coordinated, hemispheric patterns of meridional flow) it is perhaps not surprising that we found no strong link with ENSO, given that teleconnections between ENSO and the high southern latitudes tend to be localized around the southeast Pacific \citep{Simmonds1995,Turner2004}. While this result is not good news for the predictability of planetary wave activity, its increased frequency during negative SAM events offers some hope. The identified east/west migration of the mean planetary wave pattern with positive/negative phases of the SAM possibly ties in with the zonally asymmetric properties of the SAM \citep[e.g.][]{Kidson1988,Kidston2009}, however a detailed analysis of this relationship was beyond the scope of this study.

With respect to the link between planetary wave activity and regional climate variability, most relevant investigations have focused on sea ice. The recent study of \citet{Raphael2014} takes a new approach to assessing the influence of the atmospheric circulation, focusing on the ice advance (approximately March-August) and retreat (September-February) seasons for five distinct regions of sea ice variability around Antarctica. Their examination of the spatial pattern of correlation between sea ice extent and 500 hPa geopotential height for each season/region suggests that the ZW3 pattern is the primary driver of sea ice variability in the Weddell and Amundsen/Bellingshausen Seas during the advance season. Our results tend to support this finding, particularly during the early part (MAM) of the advance season. In contrast, the strong association identified between the PWI and sea ice coverage just to the north of George V Land, Ad{\'e}lie Land and the Sabrina Coast in East Antarctica does not seem to be in agreement with the results of \citet{Raphael2014}, who found the SAM to be the major driver in that region for both the advance and retreat seasons.

For the King Haakon VII Sea (10$^{\circ}$W-70$^{\circ}$E), \citet{Raphael2014} were unable to identify an obvious atmospheric driver. Our results suggest that planetary wave activity may play an important role there, since the correlation patterns identified by \citet{Raphael2014} bear some resemblance to the mean planetary wave patterns identified in this study. The reason the resemblance is not stronger may be due to the fact that the association between the PWI and sea ice coverage appears to be unidirectional in that region. In MAM, JJA and SON, PWI values greater than the 90th percentile are associated with anomalously low sea ice concentrations, while values less than the 10th percentile are associated with near average (as opposed to anomalously high) concentrations (not shown). Of course, any discussion of the atmospheric drivers of sea ice variability is complicated by the relationships between those drivers. For instance, the SAM and ENSO show many similarities in their influence on sea ice. It is unclear whether this is because they operate together in their response mechanism, or if the similarity is due to a preferred hemispheric planetary wave response \citep[e.g.][]{Pezza2012}. 

In contrast to the sea ice literature, planetary wave activity is scarcely mentioned in relation to SH precipitation variability, even in the regions of significant topography so clearly identified in this study. Instead, analyses of precipitation variability over New Zealand, Patagonia and eastern Australia tend to focus on the SAM and ENSO, with the former generally becoming increasingly influential at higher latitudes \citep[e.g.][]{Ummenhofer2007,Aravena2009,Kidston2009,Risbey2009,Garreaud2013,Jiang2013}. Such analyses may inadvertently capture some of the zonal wave influence due to its similarity with the zonally asymmetric features of the SAM, however \citet{Garreaud2013} do note that winter precipitation anomalies over Patagonia are dominated by a wavenumber three mode rather than a more zonally symmetric SAM pattern.

Planetary wave activity also receives scant attention in overviews and analyses of Antarctic temperature variability \citep[e.g.][]{Russell2010,SchneiderOkumura2012,Yu2012}. In the main, we find that the enhanced meridional flow associated with planetary wave activity brings warm air poleward and thus large positive temperature anomalies are seen throughout most of Antarctica, particularly during autumn and winter. The link between planetary wave activity and West Antarctic temperature variability is particularly interesting, given the large positive temperature trends observed in that region over recent decades \citep[e.g.][]{Bromwich2013}.  



\chapter{Introduction}

%=========================================================================

The study of the atmospheric general circulation is concerned with the dynamics of the climate system. It considers the time averaged characteristics of variables such as wind, temperature, humidity and precipitation, with the averaging applied over a period long enough to remove the random variations associated with individual weather systems, but short enough to retain monthly and seasonal variations. For a planet with a longitudinally uniform surface, the flow averaged over such a period would be the same at all points along a given latitude circle, since the average influence of zonally asymmetric transient eddies (i.e. weather disturbances) would be the same at all points. Large-scale topography and continent-ocean heating contrasts on Earth, however, provide strong forcing for zonally asymmetric planetary-scale motions in the monthly and seasonally averaged flow. These longitudinally dependent components of the general circulation may be categorised as quasi-stationary circulations, which vary relatively little in time; monsoonal circulations, which are seasonally reversing; or various subseasonal and interannual components which together account for low-frequency variability \citep{Holton2013}. In the Southern Hemisphere (SH) extratropics, the primary zonal asymmetries are the quasi-stationary zonal wavenumber one (ZW1) and zonal wavenumber three (ZW3) circulations and a mode of low frequency variability known as the Pacific-South American (PSA) pattern.

These zonally oriented waveforms span the entire longitudinal range of the hemisphere (e.g. FIG - average circulation with ZW1 and ZW3 breakdown) and are thought to owe their existence to the configuration of the continental landmasses and large-scale topography \citep{Baines1989}. When superimposed on the zonal-mean circulation, these waves produce local regions of enhanced and diminished time mean westerly winds, which strongly influence the development and propagation of transient weather disturbances. They are also associated with meridional transports of heat and moisture, which means variability in the amplitude and phase of the zonal waves is associated with monthly and seasonal variations in variables such as surface temperature and sea ice. 

Thought to be a Rossby wave response to anomalous tropical sea surface temperatures (SSTs) and consists of a wave train that arcs along an approximate great circle path from the subtropical Pacific Ocean to the southwest Atlantic (e.g. FIG - simple sketch of PSA pattern). Given this association with tropical SSTs, the PSA pattern is commonly described as the primary mechanism by which the El Ni\~{n}o Southern Oscillation (ENSO) influences the high southern latitudes. Like the zonal waves, it is an important driver of temperature and sea ice variability over and around West Antarctica and the Antarctic Peninsula. 

In light of the rapid climatic changes observed at high southern latitudes in recent decades, understanding the atmospheric drivers of regional climate variability has become an area of renewed interest. West Antarctica and the Antarctic Peninsula are among the most rapidly warming regions on Earth \citep[e.g.][]{Nicholas2014}, while Antarctic sea ice has undergone a dramatic spatial redistribution \citep{Simmonds2015}. The PSA pattern has received particular attention in relation to these changes, with some authors suggesting that it is the mechanism by which trends in tropical Pacific SSTs have been transmitted to the high latitudes.

Despite the fundamental role that the zonal waves and PSA pattern play in the SH general circulation (and perhaps recent high latitude trends), existing studies of their climatological characteristics are dated and limited in their scope. This thesis presents updated climatologies that not only utilise a longer and higher quality dataset than previous studies, but that also develop and apply new wave identification methods that fully exploit the capabilities of Fourier analysis. The results of these climatological studies have been published in a completely open and reproducible manner, and it is proposed that the approach taken in documenting the computational aspects of the work could be adopted as a minimum communication standard by academic journals in the weather and climate sciences.

###

In both hemispheres, large-scale topography and continent-ocean heating contrasts provide strong forcing for longitudinally asymmetric planetary scale time-mean motions. Such motions, usually referred to as stationary or planetary waves, are especially strong during winter and tend to have an equivalent barotropic structure, meaning the wave amplitude increases with height but phase lines tend to be vertical \citep{Holton2013}. In the context of weather and climate variability at the surface, these waves are important because they produce local regions of enhanced and diminished time-mean westerly winds, which strongly influence the development and propagation of transient weather disturbances. Persistent (or blocked) weather patterns, for instance, are typically associated with high-amplitude waves in the upper troposphere \citep[e.g.][]{Trenberth1985,Renwick2005}. The meridional transport of heat and moisture associated with these waves also influences surface conditions. 

\section{The zonal waves}

It was \citet{vanLoon1972} who first characterized SH planetary wave activity as the superposition of two zonally-oriented, quasi-stationary waveforms of wavenumber one (ZW1) and wavenumber three (ZW3). Based on Fourier decompositions of the mid-to-upper tropospheric circulation, they concluded that the net effect of the other wavenumbers was simply to modulate ZW1 and ZW3. Since that landmark study, the ZW1 and ZW3 patterns have been identified as dominant features of the mid-latitude circulation on daily \citep[e.g.][]{Kidson1988}, seasonal \citep[e.g.][]{Mo1985} and interannual \citep[e.g.][]{Karoly1989} timescales. Corresponding metrics and climatologies have been developed \citep{Raphael2004,Hobbs2007} and their relationship with circulation features including the Amundsen Sea Low \citep{Turner2013} and two prominent quasi-stationary anticyclones in the sub-Antarctic western hemisphere \citep{Hobbs2010} have been investigated.

While these climatologies and investigations reveal many of the basic characteristics of the ZW1 and ZW3 patterns (e.g. their variability and spatial pattern), with the exception of the ZW3 sea ice analyses of \citet{Raphael2007} and \citet{Yuan2008} and the ZW1 sea surface temperature results of \citet{Hobbs2007}, subsequent studies have not yet extended these climatologies to look at their influence on key variables such as surface temperature and precipitation. Related studies on topics such as Australian \citep{Frederiksen2014} and Patagonian \citep{Garreaud2013} precipitation variability sometimes mention a ZW3-like pattern in passing, but the literature lacks a broad, hemispheric perspective on the link between planetary wave activity and regional climate variability. One reason for this might be that the ZW1 and ZW3 patterns never really occur in isolation, which makes analyses of just one or the other somewhat problematic \citep{Hobbs2010}.


\section{The PSA pattern}




\section{A new wave identification methodology}

EOF limitations, Fourier advances.


\section{A practical solution to the reproducibility crisis}

Blah

\section{Thesis overview}

Blah
 

\chapter{Introduction}

%=========================================================================

The study of the atmospheric general circulation is concerned with the dynamics of the climate system. It considers the time averaged characteristics of variables such as wind, temperature, humidity and precipitation, with the averaging applied over a period long enough to remove the random variations associated with individual weather systems, but short enough to retain monthly and seasonal variations. For a planet with a longitudinally uniform surface, the flow averaged over such a period would be the same at all points along a given latitude circle, since the average influence of zonally asymmetric transient eddies (i.e. weather disturbances) would be the same at all points. Large-scale topography and continent-ocean heating contrasts on Earth, however, provide strong forcing for zonally asymmetric planetary-scale motions in the monthly and seasonally averaged flow. These longitudinally dependent components of the general circulation may be categorised as quasi-stationary circulations, which vary relatively little in time; monsoonal circulations, which are seasonally reversing; or various subseasonal and interannual components which together account for low-frequency variability \citep{Holton2013}.

Observations indicate that tropospheric quasi-stationary circulations, which are usually referred to as stationary or planetary waves, tend to have an equivalent barotropic structure. This means the wave amplitude generally increases with height, but phase lines tend to be vertical. When superimposed on the zonal-mean circulation, such waves produce local regions of enhanced and diminished time mean westerly winds, which strongly influence the development and propagation of transient weather disturbances. The most significant tropospheric quasi-stationary circulations in the Southern Hemisphere (SH) are the zonal wavenumber one (ZW1) and zonal wavenumber three (ZW3). These zonally oriented waveforms span the entire longitudinal range of the hemisphere (e.g. FIG - average circulation with ZW1 and ZW3 breakdown) and are thought to owe their existence to the topography of the Antarctic continent \citep{Baines1989}. While they exert an influence on the circulation at all times, variability in the amplitude and phase of ZW1 and ZW3 is thought to be responsible for monthly and seasonal variations in variables such as surface temperature and sea ice.

The other prominent wave-like zonal asymmetry of the SH general circulation is the Paciifc-South American (PSA) pattern. This mode of low frequency variability is thought to be a Rossby wave response to anomalous tropical sea surface temperatures (SSTs) and consists of a wave train that arcs along an approximate great circle path from the subtropical Pacific Ocean to the southwest Atlantic (e.g. FIG - PSA pattern). Given this association with tropical SSTs, the PSA pattern is commonly described as the primary mechanism by which the El Ni\~{n}o Southern Oscillation influences the high southern latitudes. Like the zonal waves, it is an important driver of temperature and sea ice variability over and around West Antarctica and the Antarctic Peninsula. In light of the rapid climatic changes observed at high southern latitudes in recent decades, understanding the atmospheric drivers of regional climate variability has become an area of renewed interest. West Antarctica and the Antartic Peninsula are among the most rapidly warming regions on Earth (REF), while Antarctic sea ice has undergone a dramatic spatial redistribution \citep{Simmonds2015}. The PSA pattern has received particular attention in relation to these changes, with some authors suggesting that it is the mechanism by which trends in tropical Pacific SSTs have been transmitted to the high latitudes.

Despite the fundamental role that the zonal waves and PSA pattern play in the SH general circulation (and perhaps recent high latitude trends), existing studies of their climatological characteristics are dated and limited in their scope. This thesis presents updated climatologies that not only utilise a longer and higher quality dataset than previous studies, but that also develop and apply new wave identification methods that fully exploit the capabilities of Fourier analysis. The results of these climatological studies have been published in a completely open and reproducible manner, and it is proposed that the approach taken in documenting the computational aspects of the work could be adopted as a minimum communication standard by academic journals in the weather and climate sciences.


\section{A new wave identification methodology}

EOF limitations, Fourier advances.

\section{A practical solution to the reproducibility crisis}

Blah

\section{Thesis overview}

Blah
 

\begin{abstract}

The major zonally asymmetric features of the Southern Hemisphere (SH) extratropical circulation are the zonal wavenumber one (ZW1), zonal wavenumber three (ZW3) and the Pacific-South American (PSA) pattern. These tropospheric waveforms play a critical role in the meridional transport of heat and moisture and in the development of blocked flow, causing the regional surface climate to vary strongly depending on the strength, frequency and phase of their activity. The PSA pattern is widely regarded as the primary mechanism by which the El Ni\~{n}o-Southern Oscillation (ENSO) influences the high southern latitudes, and in recent years it has been suggested as a mechanism by which longer-term tropical sea surface temperature trends have influenced the Antarctic climate.

This thesis presents novel approaches to identifying both the zonal waves and PSA pattern in model output. By adapting the wave envelope construct that has been recently applied in the identification of synoptic-scale Rossby wave packets, these approaches fully exploit the capabilities of Fourier analysis. They improve on existing methods by allowing for variations in both wave phase and amplitude and were applied to ERA-Interim reanalysis data in order to analyse the climatological characteristics of the waveforms and their influence on regional climate variability. The results reveal that both the zonal waves and PSA pattern are important drivers of temperature, precipitation and sea ice variability in the mid-to-high southern latitudes. While ZW1 and ZW3 are both prominent features of the climatological circulation, the defining feature of highly meridional hemispheric states is an enhancement of the ZW3 component. Identified seasonal trends towards the negative phase of the PSA pattern were largely inconsistent with recent high latitude temperature and sea ice trends. Only a weak relationship was identified between the PSA pattern and ENSO, suggesting that the pattern might be better conceptualised as preferred regional atmospheric response to various external (and internal) forcings.

In an attempt to provide a practical solution to the current reproducibility crisis in computational research, the results have been presented in a completely reproducible manner. The procedure used to document the computational aspects of the research was developed to be consistent with recommended best practices in scientific computing and seeks to minimise the time burden on authors, which has been identified as the most important barrier to publishing code. It should provide a starting point for weather and climate scientists looking to publish reproducible research, and it is proposed that relevant academic journals could adopt the procedure as a formal minimum standard. 

\end{abstract}

\clearpage
